\section{Results and Discussion}

% \begin{it}
% In many cases, the results will take the form of graphs. If you have many results, you do not need to include them all: select the ones that are most significant. This goes also if your results take the form of a calculation: it is not necessary to give all intermediate steps, but enough detail should be included to allow a person competent in the field to follow the general route taken and, if necessary, reconstruct missing steps. Try to avoid unnecessary repetition.
% \end{it}

% In the previous section, we covered the mechanics of including figures and tables into your report. We now turn to some considerations of good practice regarding your results and discussion of them. Perhaps the most obvious one of these is that figures should be displayed at a size that all text and relevant features is legible. Perhaps the next most obvious is that you should refer to every figure (and table) in the text. The usual procedure is to specify first the relevant conditions under which the results were obtained (e.g., simulation parameters, choice of experimental protocol etc), and then refer to the figure. You should make clear what is learnt by looking at the figure: that is, what is plotted as a function of what, and what the physically significant features are (e.g., is it the location of a peak? the difference between two curves? the behaviour at large or small values along one of the axes?). The caption should generally do these things to, but in a more terse manner. For example, in the case of Figure~\ref{fig:cmbr}, the key point is that the theoretical prediction (given by Planck's formula) and the observation data show an extremely close correspondence.

% Try to avoid including many similar looking figures. Often it is better to show one representative figure, and explain that other conditions yield similar results; or to plot a single quantity as a function of the relevant control parameter to show what does (or doesn't change). Remember to discuss sources of error.

% In more mathematical projects, the results may take the form of (a perhaps fairly extended calculation). Again, you do not need to include all the steps; and if you perform several calculations of the same basic type, then it may be appropriate to use the first calculation to set out the method in detail, and then skip parts of later calculations when they are essentially the same as what has already been presented.