\section{Methods}

% \begin{it}
% The methods section is a description of what you did. In an experimental project, this would comprise a description of your apparatus and the protocol you followed to obtain the measurements. In a simulation project, this would be a description of the simulation method. In a more mathematical project, there may not be any ``methods'' as such: for example, the main ideas and techniques you are developing might fall more naturally into the background / literature review section, and the application of these might be regarded as results. In this case you might decide to skip an explicit ``methods'' section.
% \end{it}

% \subsection{Structure of the document}

% The source file minimally contains:
% \begin{verbatim}
% \documentclass[12pt]{article}
% \usepackage{mphysproject}

% \begin{document}

% \end{document}
% \end{verbatim}

% If you are enrolled on the MPhys with a Year Abroad degree programme, you will be doing a shortened version of the project, for which a shorter project report is expected. If this applies to you replace the \verb|usepackage| line above with
% \begin{verbatim}
% \usepackage[yearabroad]{mphysproject}
% \end{verbatim}


% The document text goes in between the \verb|\begin{document}| and \verb|\end{document}| pair. The \verb|mphysproject| package is non-standard and has to be placed in the same folder as your source document. This package comprises the files \verb|mphysproject.sty| and \verb|PandA_black.pdf|, the latter of which is the School of Physics and Astronomy logo that appears on the title page.

% Paragraphs of text are separated by blank lines in the source document: line breaks are ignored. \LaTeX\ automatically adds an extra bit of space after full stops, which it interprets as marking the end of a sentence. Sometimes a full stop is not a full stop, e.g.\ following an abbreviation (like e.g.). You can set a normal size space by preceding it with a backslash: \verb|i.e.\ like this|.

% Sections are introduced with the command
% \begin{verbatim}
% \section{Section title}
% \end{verbatim}
% and within sections, you can have \verb|\subsection{...}| or \verb|\subsubsection{...}| if required, though you probably don't need to go any deeper than the \verb|\subsection{...}| level. 

% The \verb|mphysproject| style file redefines a few standard commands, and adds some non-standard commands to help you structure and format your document according to the rules set out in the Course Information Booklet.

% \subsubsection{Title page}

% At the start of the document (just after \verb|\begin{document}|) you should define the contents of the title page with the following commands:
% \begin{verbatim}
% \title{Title of your project} 
% \author{Your name}
% \supervisor{Your supervisor's name}
% \date{Submission date}

% \begin{abstract}
% Abstract text
% \end{abstract}

% \maketitle
% \end{verbatim}

% Repeat the \verb|supervisor| line if you have multiple supervisors, with one supervisor per \verb|\supervisor| command. The \verb|\maketitle| line at the end is essential, as without it the title page won't appear!

% \subsubsection{Personal statement}

% Before your personal statement, insert the line
% \begin{verbatim}
% \personalstatement
% \end{verbatim}
% This provides an appropriate heading.

% \subsubsection{Main text and automatic length check}

% After the personal statement and any acknowledgements, and before beginning the introduction you \emph{must} include the command
% \begin{verbatim}
% \maintext
% \end{verbatim}

% This does three things:
% \begin{enumerate}
% \item It inserts a Table of Contents.
% \item Switches the page numbering from Roman numerals to Arabic numbers.
% \item It starts counting pages for the automatic length check.
% \end{enumerate}
% If you don't include this command, three bad things happen. First, you don't get told if your report is over length. Second, Roman numerals start to look silly when the numbers get too big.  Third, it makes it hard to do the length check manually.  The pages numbered $1$, $2$, $3$, \ldots are those that are significant in terms of determining the page count for the project. These appear only if \verb|\maintext| is placed before the introduction.

% Note that \verb|\maintext| does \emph{not} start a section called `Introduction': this you must insert yourself with
% \begin{verbatim}
% \section{Introduction}
% \end{verbatim}

% The automatic length check ends either with the end of the document, or the start of the appendices, whichever comes first. If the page limit is exceeded, a warning appears in the document at this point. (It also appears in the output from \LaTeX\ that appears on the command line, although this is easy to miss).

% \subsubsection{Appendices}

% If you are including (non-assessed) appendices in your document, insert the line
% \begin{verbatim}
% \appendix
% \end{verbatim}
% at the end of your conclusion, and before the first appendix. Each appendix should be introduced with its own \verb|\section{...}| command.

% A note will be inserted into the document at the beginning of your appendices to remind you and the markers that the contents of the appendices are disregarded for the purposes of assessment.


% \subsection{Special characters and emphasis}

% \LaTeX\ was invented at a time when only a small number of characters could be reliably entered from a keyboard: therefore `special' characters have to be entered manually (i.e., using commands). These include:
% \begin{itemize}
% \item \emph{open and close quotes} --- Open quotes are entered as backticks and close quotes as apostrophes, that is \verb|`| gives ` and \verb|'| gives '. To get double quotes, double up the characters: \verb|``like this''| gives ``like this''.
% \item \emph{dashes and hyphens} --- \LaTeX\ provides three lengths of hyphens/dashes. Typing a single \verb|-| gives you a hyphen, used for hyphenating words (e.g., first-order phase transition). Typing a double \verb|--| gives you a short dash, used for ranges of numbers (e.g., 1--99). Typing a triple \verb|---| gives you a long dash---often used to introduce subclauses.
% \item \emph{accented characters} --- \LaTeX\ gives access to a large number of accents, which may be useful if you want to refer to copiously-\"{u}ml\"{a}\"{u}t\"{e}d heavy-metal bands, for example. Do a Google search on `latex accents' to see how to get them.
% \end{itemize}

% The two most common types of emphasis you might wish to employ are \textit{italics} and \textbf{bold face}, which were set with \verb|\textit{italics} and \textbf{bold face}|. Purists prefer the \verb|\emph{...}| command, as this expresses the notion that it is emphasis that you want, rather than bold or italics specifically. \LaTeX\ will choose an appropriate emphasis, which is usually italics.

% \subsection{Equations}

% The reason that \LaTeX\ has established itself as \emph{the} typesetting system for physicists and mathematicians is that it allows complex equations to be included in your document with ease. Whole books have been written on this subject. However, about $95\%$ of your needs are provided for by the following tips.

% \begin{itemize}
% \item There are two (main) ways to include equations. The first is to include them inline. For example, Newton's second law can be written as $F = ma$, which is typeset as \verb|$F = ma$|. A single dollar sign opens and closes an inline equation. The second type of equation is a displayed equation. For example, the partition function of the quantum mechanical harmonic oscillator is
% \begin{equation}
% \label{Z1}
% Z(1) = \frac{{\rm e}^{-\frac{\hbar\omega}{2kT}}}{1 - {\rm e}^{-\frac{\hbar\omega}{kT}}} \;.
% \end{equation}
% In the source document, this equation is written as
% \begin{verbatim}
% \begin{equation}
% Z(1) = \frac{
%   {\rm e}^{-\frac{\hbar\omega}{2kT}}}
% }{
%   1 - {\rm e}^{-\frac{\hbar\omega}{kT}}}
% } \;.
% \end{equation}
% \end{verbatim}
% There are many mathematical symbols you can include: tables of these symbols can be found by doing a Google search on `latex math symbols'. Sometimes you need to include additional packages (like \texttt{amsmath} or \texttt{amssymb}) to use more exotic symbols.

% Try and use the right symbols where you can. For example, a common error is to use greater-than and less-than signs instead of angle brackets (\verb|\langle| and \verb|\rangle|). Good: $\langle E \rangle$ and $|\psi\rangle$. Bad: $< E >$ and $|\psi>$.
% \item If you have brackets around sums, integrals, fractions or other large objects, use \verb|\left| and \verb|\right| so that they are automatically made large enough to enclose their contents. Compare for example
% \begin{equation}
% \left( \sum_{k=0}^{\infty} z^k \right) \;,
% \end{equation}
% typeset as \verb|\left( \sum_{k=0}^{\infty} z^k \right)|, as opposed to
% \begin{equation}
% ( \sum_{k=0}^{\infty} z^k ) \;,
% \end{equation}
% which does not use \verb|\left| or \verb|\right|.
% \item The mysterious error `You cannot use \verb|\eqno| in math mode' means that you have forgotten a \verb|\right| that is needed to match a \verb|\left|.
% \item Always use the \verb|\begin{equation}| \ldots \verb|\end{equation}| environment for displayed equations, rather than one of the other options available to you. This makes sure that an equation number appears. Even though you may not refer to a particular equation, someone else might want to!
% \item Equations should be punctuated as normal text. So if a displayed equation ends a sentence, place a full stop at the end of it. It helps to insert a little bit of spacing, using \verb|\;| before this punctuation.
% \item It is tempting in the source to insert blank lines before and after equations for readability. Do not do this, as this starts a new paragraph (inserting extra space) each time. Only put a blank line if the paragraph has genuinely ended.
% \item Some people insist that exponential ${\rm e}$ and total derivatives ${\rm d}$ should be set in roman font using \verb|{\rm e}| and \verb|{\rm d}| whilst others find this overly fussy.
% \item However, everyone agrees that you should set things like $\sin x$ and $\cos x$ using \verb|\sin x| and \verb|\cos x| and not as \verb|sin x| and \verb|cos x|, as the latter come out as $sin x$ and $cos x$: that is, these read as the variable $s$ times $i$ times $n$ times $x$ and as $c$ times $o$ times $s$ times $x$, and not what you actually want.
% \end{itemize}

% \subsection{Figures and tables}

% The \verb|mphysproject| package includes a set of commands (inherited from the \verb|graphicx| package) that allow you to include figures in PDF, PNG and JPEG formats directly into the report. See Figure~\ref{fig:cmbr} for an example, where the original image lies in a file called \verb|Cmbr.pdf|.

% A figure is a special kind of object called a \emph{float} that \LaTeX\ positions separately to the text, usually somewhere near the point at which it is embedded in the source code. The typical sequence of commands is
% \begin{verbatim}
% \begin{figure}[htb]                
% \begin{center}
% \includegraphics[width=10cm]{filename}
% \end{center}
% \caption{\label{reference_label} Caption goes here.}
% \end{figure}
% \end{verbatim}

% The \verb|\begin{figure}| \ldots \verb|\end{figure}| pair encloses the floating material that can be reposition. The referenced external file should lie in the same folder as the \LaTeX\ source document. You can leave the extension (\verb|.pdf| etc) off the filename, and \LaTeX\ should still find it. See Section~\ref{sec:ref} below for more information about the reference label.

% \begin{figure}[htb]                % h = here; t = top (of page); b = bottom (of page)
% \begin{center}
% \includegraphics[width=10cm]{Cmbr} % Vary the value in the width=XX setting to resize the figure if need be.
% \end{center}
% \caption{\label{fig:cmbr} The cosmic microwave background radiation spectrum, adapted from the Wikipedia page \cite{Cmbr}. The line is Planck's formula, and the crosses are observational data (errors are smaller than symbol size). Note the close correspondence between the two.}
% \end{figure}

% When starting out with \LaTeX\ it is natural to attempt to wrest some control over the placement of figures. Let me save you several months of your life at the outset, and recommend you just let \LaTeX\ get on with it. Your main control over the display of the figure lies in where the \verb|\begin{figure}| \ldots \verb|\end{figure}| block appears in the text. If the figure comes too soon, move it later; if it comes too late, move it earlier. The three characters \verb|htb| comprise a set of polite requests as to where you want the figure to go: \verb|h| means `here', \verb|t| means `at the top of a page' and \verb|b| means `at the bottom of a page'. We say \emph{a} page (not \emph{this} page) advisedly. You can add an exclamation mark \verb|!| to make a less polite request; \LaTeX\ may nevertheless rudely ignore it. You can adjust the size of the figure by changing \verb|width=10cm| to something else. You can put two figures side by side above the same caption by repeating the \verb|\includegraphics| line (and adjusting the filename appropriately).

% A similar construction, enclosed by a \verb|\begin{table}| \ldots \verb|\end{table}| pair is available for the inclusion of tables. Again you can add the \verb|htb| request to exert some influence over where it appears. See Table~\ref{tab:atable} for an example. You are referred to the source, and in particular \cite{Kopka2003} who are unusually enthusiastic about tables, for information about how the tabulated content is marked up in \LaTeX.

% \begin{table}[htb]
% \begin{center}
% \begin{tabular}{l|l|l} % This specifies three left-justified columns separated by vertical lines. Change l to r or c to get right-justified or centred columns; add more columns if you need them
% Particle type & Spin & Multiple occupancy allowed \\\hline\hline % Columns are separated by &, rows by \\
% Bosons & Integer & Yes \\
% Fermions & Half-integer & No \\\hline % An \hline after a row separator \\ puts a horizontal line after the row
% \end{tabular}
% \end{center}
% \caption{\label{tab:atable} Summary of the key properties of fermions and bosons.}
% \end{table}


% \subsection{Cross-referencing within the document}
% \label{sec:ref}

% The second most useful feature offered by \LaTeX\ (after its unparalleled equation typesetting abilities) is being able to label numbered items in the text, and refer to them later by that label. This is useful because if you reorganise the text (e.g., add extra figures) which causes items to be renumbered, all the cross-references automatically update.

% To create a label, insert \verb|\label{reference_label}| in the text. To refer to the labelled item, insert \verb|\ref{reference_label}| in the text. For example, this current subsection, numbered \ref{sec:ref}, is labelled \verb|sec:ref|. There is an apparently unwritten but widely-used convention of `namespacing' the labels so that section labels all begin with \verb|sec:|, figures with \verb|fig:| etc, but you don't have to do this.

% \LaTeX\ is pretty smart about knowing \emph{what} is being labelled. If you place the label among the regular text, it will refer to the current section. If you place it inside a \verb|\begin{equation}| \ldots \verb|\end{equation}| pair, it will label the equation. It seems figures and tables are mostly reliably labelled if you include the \verb|\label| inside the caption. Alas \LaTeX\ is less smart about referring to the label; for example, if you want to refer to an equation, you have to manually type the brackets around the equation number: \verb|(\ref{Z1})| gives (\ref{Z1}); likewise you will need to include the word `Figure' when referring to Figure~\ref{fig:cmbr}: \verb|Figure~\ref{fig:cmbr}|. (Note the \verb|~| here stops the word Figure and the figure number appearing on different lines of text, which some people find unappealing).

% It is worth having a little understanding of how the cross-referencing system works under the hood. As your source file is compiled, \LaTeX\ creates a file with the extension \verb|.aux| that records what each label refers to. This auxiliary file is read in when compilation starts. What this means is that when you \emph{first} compile the document, it won't actually know what a label refers to, as this information isn't in the auxiliary file yet. As this point, references appear as question marks (?). So you have to compile it again to get the references to appear correctly. Very occasionally, you might need to compile a third time, as sometimes the formatting of the document can change due to the correct label text being inserted. This goes particularly for the table of contents, as page numbers might change with the formatting. Before submitting your report, it is usually a good idea to make sure you have run \LaTeX\ multiple times, and look out very carefully for warnings about missing references in the log that it produces.

% \subsection{Referencing external sources}

% A similar command is provided for citing papers, textbooks, webpages etc. Each item in the list of references has a label (also called a key), and a citation is inserted in the text with the command \verb|\cite{label}|. For example, \verb|\cite{Kopka2003}| gives the citation \cite{Kopka2003}. If you want to cite multiple works at the same point in the text, do it like this: \verb|\cite{Kopka2003,Cmbr,Askin1986}| which appears as \cite{Kopka2003,Cmbr,Ashkin1986}. Treat the citation as a word in the text: that is, it should be preceded by a space, and followed either by a space or a piece of punctuation.

% We defer a discussion of how the reference list is constructed, and its items labelled, to the References section below (Section~\ref{sec:refs}).

Having now laid out the fundamentals, we investigate the implementation of generative machine learning techniques to explore the classification of knots using segment-to-segment writhe.

\hl{Considerations for designing the model architecture}:

    1. Data
   - Data Representation: How is the knot data represented. Design input layer to accommodate this representation.

   - Data Complexity: Consider complexity of knot data. Some knots might have simple structures, while others can be highly intricate. Model architecture should be variable to capture this complexity.

    2. Generative Model:
   - GAN -> generating realistic data.
   - VAE -> learning a probabilistic distribution of the data.

   - Hybrid Model -> Hybrid approach to combine both VAE and GAN (will need to research).

    3. Encoder and Decoder Architecture:
   - Encoder: (VAE or hybrid) -> encoder maps input knot data to a latent space representation. 
   Convolutional layers work well for \emph{spatial} features.

   - Decoder: decoder maps latent space back to the original data space. 
   Important! -> output layer matches input.

    4. Loss Function:
   - Adversarial Loss: (GAN). Coupled with a discriminator that distinguishes between real and generated knots.

   - Reconstruction Loss: (VAE). Helps model learn underlying distribution of data.

    5. Considerations for Knot Theory:
   - Topological Features: Ensure that model architecture can capture and generate features.

   - Invariant Representations: Make model invariant to certain transformations that don't change the underlying knot structure. eg, chirality + mirroring.

    6. Hyperparameter Tuning:
   - Learning Rate (test).

   - Batch Size (test).

   - Architecture Depth and Width: (test) depth and width of nn layers.

    7. Regularization and Normalization:
   - Dropout layers: prevent overfitting.

   - Batch Normalization: Normalize inputs to each layer -> faster convergence and better generalization.

    8. Validation and Testing:
   - Validation Set: check OG code.

   - Early Stopping: check OG.

    9. Visualization:
   - Visualization: PCA and t-SNE visualize latent space and see how model is separating different types of knots.

