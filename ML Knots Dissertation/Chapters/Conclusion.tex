\section{Conclusion}

% \begin{it}
% In the conclusion you should bring together your main findings, and explain what we have learnt from them collectively. You should refer back to the broader research area and outstanding questions alluded to in your introduction, and make clear what progress has been made towards understanding them as a result of the project. Suggestions for further work should also go here.
% \end{it}

% In this template document, we have provided a very brief introduction to creating a project report using \LaTeX, with particular emphasis on the special features of the \verb|mphysproject| package that simplify the process of creating a report that is compatible with the requirements for the MPhys Project course. As has been stated on a number of occasions, we are only able to scratch the surface of what is possible, although it should cover the vast majority of your needs. You are reminded that this document includes only a precis of the guidance that is available on the MPhys Wiki, links to which were provided in the Introduction.
