\section{Background}

% \begin{it}
% The background section is where you set out, in a self-contained way, the key concepts that it is essential for the reader to understand in order to make sense of the results. This would also be an appropriate place to discuss other attempts in the literature to solve the problem you are faced with, and critique the successes and limitations of these. By the end of this section, the reader should have some understanding of what would count as a successful outcome from the project.
% \end{it}

% \LaTeX\ is the standard software for typesetting technical documents (like research papers and project reports) in the mathematical and physical sciences (although it is increasingly used in other fields as well). Unlike other document preparation software that you might be familiar with, like Microsoft Word, you do not write directly onto the page. Instead, you write your text a \emph{source file} that contains regular text (like this) as well as a set of commands that allow other material---notably figures and equations---to be inserted.  These commands are introduced with a backslash character; arguments to these commands are placed inside pairs of curly brackets.

% For example, to emphasise text, you use the command \verb|\emph{emphasised text goes here}| which renders as \emph{emphasised text goes here}. Throughout this document, we will use \verb|this font| to denote \LaTeX\ source code (or commands to be entered at the command line, see below).

% The process of converting your \LaTeX\ source to a readable document is referred to as \emph{typesetting} or \emph{compiling} your document (the latter in analogy to compiling human-readable source code in a language like C or C++ to a code that can be executed by the computer). How this is done depends on the environment you are writing your source code in:
% \begin{itemize}
% \item The classical approach is to edit your source in a text editor, like \verb|emacs| or \verb|vi| (if you are super old-school), and run the typesetting engine from the (Linux, MacOS or Windows) command line. Usually the program that you will want to run is called \verb|pdflatex|, which can incorporate figures in PDF, PNG and JPEG format, and outputs a PDF document. If your sourcefile is called \verb|darkmattereport.tex| it is run like this (\verb|$| denotes a shell command prompt):

% \begin{verbatim}
% $ pdflatex darkmatterreport
% \end{verbatim}

% This will generate typically a lot of output, and leave a file called \verb|darkmatterreport.pdf| in the same folder (directory) as the source file. You can look at this in a PDF viewer, like Adobe Reader. Modern PDF viewers should be able to detect when the document has been updated, and refresh automatically.

% This approach will work in the CPLab, and most likely also on your own computer, whether Linux, MacOS or Windows based.
% %
% \item The more modern approach is to use an dedicated \LaTeX\ editor, which will usually have buttons or keyboard commands that compile and display the output for you. The standard \LaTeX\ distribution for macOS users is \href{https://tug.org/mactex/}{Mac\TeX} which comes with an app called \TeX Shop (if you install the GUI apps alongside): this is a very user-friendly editor. The standard distribution for Windows users is \href{https://www.tug.org/texlive/}{\TeX\ live} and comes with the \TeX works editor. \TeX works also runs on macOS and Linux.
% %
% \item An increasingly popular alternative to a desktop app is to use a web-based \LaTeX\ editing service, like \href{https://www.overleaf.com}{Overleaf} or \href{https://www.sharelatex.com}{Share\LaTeX}. One advantage of these is that you do not have to install \LaTeX\ on your own machine (which has quite a hefty footprint), and you will get the same view of the document from different machines (e.g., Linux hosts in the CPLab, Windows-based machines in the library, your own laptop etc). The downside is that you do need an internet connection, and compilation can be slow.
% \end{itemize}

% Fortunately, \LaTeX\ is very mature and quite standardised, so you should find your documents can be ported easily between different environments if required. The main issue is whether any packages you use are installed, but unless you use extremely obscure packages, this is unlikely to be an issue.

% In the next section we outline some of the key features of \LaTeX\ and most commonly-used commands. However, \LaTeX\ is a vast, highly configurable system that we cannot describe exhaustively in such a short document. Instead we refer you to the following places for more information:
% \begin{itemize}
% \item The \href{https://www.wiki.ed.ac.uk/x/7y6UCw}{source code} to this document, which will allow you to see how it was created!
% %
% \item \href{https://www2.ph.ed.ac.uk/~wjh/tex/ilw/slides-beamer.pdf}{Will Hossack's Short Introduction to \LaTeX}.
% %
% \item \href{https://www.latex-tutorial.com/tutorials/}{An online step-by-step guide to \LaTeX}.
% %
% \item The book \textit{A Guide to \LaTeX} by Kopka and Doly \cite{Kopka2003}.
% %
% \item The \href{https://tex.stackexchange.com}{\TeX-\LaTeX\ stack exchange}, a board where questions and answers on aspects of \LaTeX\ (and a lower-level system called \TeX\ that it is built on) are posted.
% %
% \item A search engine known as \href{https://www.google.com/}{Google}, which you may have heard of.
% \end{itemize}

\subsection{Knot theory}
\subsubsection{Knot classification}

\subsection{Machine learning}
\subsubsection{The classification problem}
\subsubsection{Neural networks}

\subsection{Generative Machine learning}
\subsubsection{The Autoencoder}
\subsubsection{GAN's and VAE's}
\hl{Examples}